\documentclass[russian]{vegareport}

\usepackage{lipsum}
\addbibresource{black76.bib}

\title{Влияние макроэкономических и социальных показателей на уровень счастья}
\author{Сулейманов Асхаб, Заостровский Всеволод, Черепахин Иван}
\date{}
\usepackage[english,russian]{babel}

\begin{document}
    \maketitle

    \chapter{Введение}
        \section{Слова и философия}
        В чём смысл существования государства? Этим вопросом задавались тысячи мыслителей на протяжении всей истории человечества. Более того, очень многие из них были твёрдо убеждены в том, что именно им удалось услышать голос Истины, хотя, конечно, сценарий при котором одна и та же Истина, вмещающая в себя сущность госудерства, одновременно описывается миллионами эпитетов, аллегорий и фразеологизмов, колеблющихся от "самого холодного из всех холодных чудовищ" до "намордника для усмирения плотоядного животного, называющегося человеком",  кажется, по меньшей мере, весьма и весьма маловероятным. Мы, конечно, не ставим задачи углубиться в тысячелетнюю историю этой дискуссии, и отвечаем на этот вопрос до наивности безхитростно: смысл государства, равно как и всякого изобретения человечества, состоит в привнесении счастья в печальное настоящее нашего рода. 
        \\
        Однако, далеко не все творения покорно следуют замыслам своих создателей, потому интересно разобраться в том, каких побед человечеству удалось достичь на этом поприще и в каком направлении следует двигаться дальше.
        \\
        Поиски ответа на эти вопросы заставляют задуматься, по меньшей мере, о том, как измерить это самое счастье. Существуют очень большие и серьёзные группы исследователей, обеспокоенных этим вопросом. Сайт World Happiness Report встречает своих читателей словами "our success as countries should be judged by the happiness of our people". В рамках работы, именно \href{https://worldhappiness.report/ed/2017/}{эти} данные мы будем использовать в качестве меры счастья.
        \\
        Следующая вопрос, ответ на который необходимо сформулировать, такой: "как характеризовать государство и направление в котором оно движется?" Многообразие путей, которые можно избрать для исследования этого аспекта пугающе велико. Есть существенные основания полагать, что счастье имеет социокультурный генезис, поэтому возникает искушение углубиться в этот вопрос в рамках психологической или филосойской парадигмы. Тем не менее, сушественным недостатком этого подхода является сложность или невозможность формализации и дальнейшего количественного исследования результатов. Мы будем рассматривать исключительно объективные и измеримые показатели, по которым принято судить о развитости экономики государства, а также те, что могут существенно влиять на самоощущение его граждан.

        \section{Модель не эндогенна (но это не точно)}
        Рассматриваются следующие объясняющие переменные:
        \begin{enumerate}
            \item \href{https://www.kaggle.com/datasets/mvieira101/global-cost-of-living}{Индекс цены жизни}
            \item \href{https://data.worldbank.org/indicator/GB.XPD.RSDV.GD.ZS?view=chart}{Уровень безработицы.}
            \item \href{https://data.worldbank.org/indicator/NY.GDP.PCAP.PP.CD}{ВВП по ППС на душу населения.}
            \item \href{https://data.worldbank.org/indicator/MS.MIL.XPND.GD.ZS}{Военные расходы.}
            \item \href{https://data.worldbank.org/indicator/FP.CPI.TOTL.ZG?view=chart}{Инфляция.}
            \item \href{https://data.worldbank.org/indicator/NY.GNS.ICTR.ZS?view=chart}{Накопления физических лиц.}
            \item \href{https://data.worldbank.org/indicator/IC.TAX.TOTL.CP.ZS?view=chart}{Налоговая нагрузка физических лиц.}
            \item \href{}{Число зарегестрированных патентов по отношению к населению государства.}
            \item \href{https://data.worldbank.org/indicator/SP.POP.DPND}{Демографическая нагрузка.}
            \item \href{https://data.worldbank.org/indicator/SH.XPD.CHEX.GD.ZS}{Расходы на медицину.}
            \item \href{https://data.worldbank.org/indicator/SE.XPD.TOTL.GD.ZS?view=chart }{Расходы на образование.}
            \item \href{https://data.worldbank.org/indicator/VC.IHR.PSRC.P5}{Число убийств на тысячу человек.}
        \end{enumerate}

    \chapter{Результаты}
        
    \chapter{Выводы}


\end{document}